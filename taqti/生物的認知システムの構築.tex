\documentclass{jsarticle}
\setcounter{secnumdepth}{5}
\setcounter{tocdepth}{1}
\usepackage[dvipdfmx]{graphicx}
\usepackage[dvipdfmx]{xcolor}
\usepackage{amsmath,amssymb,amsthm,epic,eepic,multicol,amscd,enumerate,framed,tikz, siunitx}
\usepackage{bm}
\usepackage{lipsum}
\usepackage{wrapfig}
\usepackage[version=3]{mhchem}
\usetikzlibrary{cd}
\usepackage[dvipdfmx]{hyperref}
\usepackage{pxjahyper}
\hypersetup{% hyperrefオプションリスト
setpagesize=false,
 bookmarksnumbered=true,%
 bookmarksopen=true,%
 colorlinks=true,%
 linkcolor=blue,
 citecolor=red,
}
\setlength{\evensidemargin}{\oddsidemargin}
\newtheoremstyle{break}
  {\topsep}{\topsep}%
  {\itshape}{}%
  {\bfseries}{}%
  {\newline}{}%
\theoremstyle{break}
\newtheorem{bigthm}{大定理}
\newtheorem{thm}{定理}[section]
\newtheorem{prop}[thm]{命題}
\newtheorem{lem}[thm]{補題}
\newtheorem{fact}[thm]{事実}
\newtheorem{cor}[thm]{系}
\newtheorem{situ}[thm]{状況}
\newtheorem{defn}[thm]{定義}
\newtheorem{prob}[thm]{問題}
\newtheorem{ob}{観察}
\newtheorem{note}[thm]{注意}
\newtheorem{ex}[thm]{例}
\newtheorem*{axiom}{公理}
\newtheorem{opi}[ob]{私見}
\usepackage[utf8]{inputenc}

\title{生物的認知システムの構築}
\author{}
\date{2024/3/20 20:52}
\begin{document}
\maketitle

\section{negativity, positivity, flat}
positivity, negativity に関する事項についてふれておく。positivity が付与されている空間とは、そこにおいては「閉じており、コンパクトであり、有界であり、有限性・限定性があり、すべての物事が最終的な末路として影響し合う混沌となる」ようなものであることが理解される。negativity が付与されている空間とは、そこにおいては逆に「開いており、非コンパクトであり、非有界であり、いかなる状況においても新たなものごとを書き込むことができる」ようなものであることが理解される。あるいは、flat な状況というのは、「非コンパクトであり、また展延可能であり、簡単に貼り合わせることができる」ような状況であると理解することができるだろう。

もちろん、我々があくまでも有限的な世界にいるという要件のもとでは、「最終的なレベルにおいて」positivity による制約を受けることについては、もちろん否定の余地はない - しかしながら、positivity が最終的に要求されるスケールを変更することが可能であることについては着目しておきたい - これは紙の原理に代表される「実効無限の構築」なる方法論に観察される。

\begin{ex}
紙あるいは記号の導入によって、われわれは物事 - 情報の記憶にあたって、生物学的な (認知的な) 容量によって制約される positivity を脱却し、「紙の量」という - スケールの完全に異なる - 量に positivity を帰着させることができる。「紙」というフレームワークがいかにして実現されているかについても観察をしていきたい - すなわち、我々の記憶・認知の方法と、言語・記号を対応させる - コーディングをおこなう。そのコーディングの構築により、「認知物」を「言語物」に変換することに (一定のレベルで) 成功し、そして言語物についてはこれを物理的媒体に永続的に保存することが可能である。言語はなぜ可能になったか - これはまさに分化を記録する局所 negative な認知の方法が可能であったことに由来する - 脳構造には negativity が可能な側面がある。これは、network の理論にも通ずるかもしれない。network の数学的理論については、探究が望まれる。
\end{ex}


\section{問題解決の方法}
数学的な、個人的な経験のもとでは、充分に negativity を発達させることによって文明を構築し、それを身体にすることによって状況を理解し、最終的に positivity - ad hoc な方法によって解決する。このような認識・認知の方法によって具体的事項に関する理解が実行される。あるいはこのような方法によって到達可能なものに私の興味が比較的偏っていると記述するべきかもしれない。

negativity を発展させることによって、positivity のスケールを変更する - このようなことは可能であり、結局のところ positivity という制約はあるわけだが、局所的に negative な身体であることを思い出すことによって、positivity のもとでも対処可能であるようにする - このようなナラティブであると理解される。

\section{紙の原理にみられる身体の動性}
紙の原理は本質的に「動性」を前提している - すなわち、復号・復元に関する作業を要求し、またその作業がうまくいくであろうという仮説・希望のもとで (あるいはすべて覚え続けるという「うまくいきすぎる理想」を脱却することによって)、実効無限を手にいれることができる、というストーリーあるいはナラティブである。

\section{紙の原理にみられる negativity - flatness}
言語とは本質的にその再帰性は negativity あるいは flatness を要求する - このことによって、本質的に positive な部分が棄却されてしまい、あるいは比較的無視されてしまう - そのために、思考そのものについて positivity が無視される - そのような「不具合・無理」が生じてしまう可能性がある。このような動性があることについても理解しておきたい - positivity を保存することは、そのナラティブをできる限り記述し、各々の身体に任せる必要がある - すなわち、「武芸」の伝達としての「稽古」が発生することに任せなければならない。

あるいは、ミームとは、ある種の生物学的 - ウイルス的な紙の原理である - これはより大きな positivity のなかに - あるいはより大きなスケールのなかに - 情報を復元するカギを埋め込む行為であるともいえる。positivity の記憶装置として - 人を用いる。この意味で他人は紙でもある。

\section{食虫植物のモチーフ}
その positivity - negativity を変えながら、植物という枠組みを超えて「虫」というタンパク源を「捕食」することを可能とする。これはある種に 2 節にみられる問題解決の方法と符号する。あるいはそのような認知の方法についての可能性が理解される。

\section{休息}
認知におけるインプット - 反応は現在酷使されている - と解釈することも可能であろう。均一に酷使される身体ではなく (不随意筋?)、これを管理者権限のもとに変動させるといかなることがおこるか?

技術的な予想としては曲率変動を大きく、身体の柔軟性をあげ、あるいはまたストレスを軽減することが可能であると理解される。しかしながら、本質的な予想についてはなにももっていない。

\section{生物的認知システムの可能性}
constructive な発想においては、認知システムについてもこれは構築物である。ミームと脳の初期設定から復号された認知システムを現在もっているが、しかしながら充分 ampleness の保証された状況においては、(あるいは適切な死・破滅の可能性のなか) 可能なものはすべて実現される。イキモノは可能なシステムである - ミームもまたイキモノ的な性質をもつ - この観点において認知システムとしてイキモノ的な発想を導入できるか?このイキモノの生きる環境における重要な言語として positivity, negativity のような曲率が色としてある。しかしながら、他にも色 - 音 - 地形はある可能性があり、それらのなかに生きる「イキモノ」の博物学が要求されている。


\end{document}
















