\documentclass{jsarticle}
\setcounter{secnumdepth}{5}
\setcounter{tocdepth}{1}
\usepackage[dvipdfmx]{graphicx}
\usepackage[dvipdfmx]{xcolor}
\usepackage{amsmath,amssymb,amsthm,epic,eepic,multicol,amscd,enumerate,framed,tikz, siunitx}
\usepackage{bm}
\usepackage{lipsum}
\usepackage{wrapfig}
\usepackage[version=3]{mhchem}
\usetikzlibrary{cd}
\usepackage[dvipdfmx]{hyperref}
\usepackage{pxjahyper}
\hypersetup{% hyperrefオプションリスト
setpagesize=false,
 bookmarksnumbered=true,%
 bookmarksopen=true,%
 colorlinks=true,%
 linkcolor=blue,
 citecolor=red,
}
\setlength{\evensidemargin}{\oddsidemargin}
\newtheoremstyle{break}
  {\topsep}{\topsep}%
  {\itshape}{}%
  {\bfseries}{}%
  {\newline}{}%
\theoremstyle{break}
\newtheorem{bigthm}{大定理}
\newtheorem{thm}{定理}[section]
\newtheorem{prop}[thm]{命題}
\newtheorem{lem}[thm]{補題}
\newtheorem{fact}[thm]{事実}
\newtheorem{cor}[thm]{系}
\newtheorem{situ}[thm]{状況}
\newtheorem{defn}[thm]{定義}
\newtheorem{prob}[thm]{問題}
\newtheorem{ob}{観察}
\newtheorem{note}[thm]{注意}
\newtheorem{ex}[thm]{例}
\newtheorem*{axiom}{公理}
\newtheorem{opi}[ob]{私見}
\usepackage[utf8]{inputenc}

\title{呪術探究 20240623}
\author{}
\date{}
\begin{document}
\maketitle

ここ最近の呪術的探究の進展について、非体系的なかたちで記述を試みる。

\section{多体系}

呪術とは結局「意志の実現」に関する体系であると (いまの素朴な観点においては) 理解されるが、そこにおいても「もののことわり」は重要であり、さらにいえば諸事物が (ひろくいえば) いくつかの数学的諸原理によって解釈、あるいは理解される。

この viewpoint において、「認知システム」の力学を記述することは、ひとつの重要な「プレ・呪術」となるようにおもわれる。しばしば「内的世界の探究」として語られることも (古典的な呪術、あるいはその修練の段においては) 多かったのではないかと憶測する。認知体のひとつの重要な性質として、それ自身がひとつの「多体系」であることが挙げられる。もちろん「認知システム特有の物性」は「多体系である」ということのみに由来するものではないが、しかしながら「多体系」の一般論に注目することは、より精密な力学の把握をおこなうためにも重要であろう。

また内的探究にかぎらず、「多体系」は現世環境のさまざまな部分に表出する。流体、経済、社会通念、トレンド、こういったものはある種に「多体系の力学」の結果として (少なくとも定性的には) 理解される現象を引き起こすようにおもわれる。もちろん、「多体系の一般論」を定量的なレベルで構築しようというのは、ある意味では「万物に対する知識を定量的な形で記述しよう」ということであるため、これは非常に困難な課題である。しかしながら、定性的なレベルで、ある程度の記述をおこなうことは可能であると考える。以下、そのような指針のもと、「多体系」にしばしばみられる「現象」を記述しようとおもう。

\begin{note}
ここに述べた「定量的記述」と「定性的記述」について、具体的な例のもとでその「ちがい」を確認したい。「多体系」のもたらす現象のひとつに、「天気」が挙げられる。「天気」の予測の複雑性については、しばしば「カオス」の言葉で語られる。したがって、その「カオス」の素朴な哲学、あるいは説明に則れば、「天気は些細なきっかけでおおきく変動することがある」- このことは理解される。これはある意味で「天気に関する性質の定性的な記述」である。しかしながらこの定性的理解は、「どのような小さなきっかけによって、天気の結果がどのようにおおきく変動するか」という種類の「具体的な結果」をもたらすことは基本的にはない。まさに「定量的記述」とは、「天気予報」にほかならないわけであるが、現在「天気予報」は高性能な計算機によって、観測された気象情報をもとに「確率的」あるいは「統計的」な方法によって実行されている - と私は理解している。
\end{note}

多体系にしばしばみられる現象のひとつに、(認知的 - 把握的なレベルにおいての)「実質的なパラメータ数が驚異的な低下している」というものが挙げられる。

\begin{ex}
現世物理的な例でいえば、気体等を解釈するさいの熱力学あるいは統計力学などは、まさにそのような「実質的な (少ない) パラメータ」を取り扱う体系であるといえるだろう。気体というのはこれも多体系であり、しかしながらその大域的な挙動 (認知のレベルにおける実質的な挙動) は、たとえば「温度」「圧力」「内部エネルギー」「粘性」「質量」「気体の総量」「体積」といった「大雑把な量」によって統制される - というところに、「多体系の力学の驚異的な部分」があらわれているともいえるだろう。
\end{ex}

多体系の挙動を観察するにあたって、重要なこととして、「ひとつひとつの粒子」の動きを個別に追うことなく、大域的な振る舞いを「ある程度」理解することが「しばしば可能となる」ことにある。ただし、ひとつのポイントとして、このような「大域的制御」を可能とするためには、たとえば「おなじような物性をもつ粒子がたくさん集まっている」ということも重要であろう。認知機能についても、「脳細胞」という「粒子」の物性、そしてその「集積」によって引き起こされる大域的な挙動としての「生命機能」あるいは「認知機能」、さらにいえば「知性」がひきおこされるのではないか、と憶測することもできる。

\begin{ex}
局所的な粒子の物性が大域的な挙動を決める - ということについては、たとえば微分方程式論について考えれば直感的にもよくわかるだろう。微分方程式論というのは、関数の局所的な挙動のみを法則として与えたとき、関数の大域的な挙動がどのようになっているかを調べる学問である。微分方程式の基本的な例のひとつに、熱方程式というものがある。これはすなわち、「関数の値 (その時点での温度) と、関数のラプラシアン - 発散の具合 (熱の放散の具合) が比例する」という法則であり、このような法則をみたす関数は、(エネルギー条件のもとで) 熱は無限遠に放散しつづけ、やがて $0$ に向かう - ということが理解される。
\end{ex}

なぜこのようなことが可能になるのか - ということについて少し考えてみたい。たとえば、「おたがいの粒子がおたがいの運動を増幅させる」ような系においては、逆説的にこのような「大域的把握」はむずかしいであろう。実際、粒子の個性が増幅しあうならば、ひとつの粒子の局所的な運動が大域的な部分へも、ほぼ常におおきく影響することとなるが、そのような系については (本稿の意味での)「熱力学」は不可能である。したがって、ある意味で「熱力学」が可能となるためには、「粒子の統計的な個性は相互関連のもとで不干渉的あるいは減衰しあう」ことが要求される - ようにおもえる。

そもそも「大域的把握」とはどのようなことをいうのか ? 状態とはなにか ? この問いについて、より探究をすすめる必要があるように感じる - 憶測のレベルでおもいうかぶものとして、Noether の定理 - 系の対称性が保存量の存在をみちびく - がある。このような「もののことわり」によって、すなわち「均質ななにか」のうえには、その均質性の分だけ「大域的な把握」が可能となるのではないか。この均質性は、まさに「多体系」であることゆえに担保されるものであるとも解釈できる - すれば、なにかが「多体系であること」それ自体が驚異的なことであるとも理解できる。

なにかの「ことわり」によって、「多体系」が表出してしまうと、その系が「あまりの粒子の多さ」ゆえに、「統計的な意味で均質となってしまい」、ゆえに「対称性」が生じて - したがって Noether の定理によって「保存量」の存在が理解される。われわれの「認知機構」は、基本的にこのような「マクロ・データ」に対処したいわけであって、目的論のレベルでいえば、このような「マクロ・データ」の関連にのみ興味がある - ミクロにおける物性、物理学の「マクロ化」の結果として得られる体系こそが「熱力学」なのではないか ?

なぜこのようなマクロの概念があらわれるのか - ということの根底に、「分化現象」があるように感じられる。そのため、このことを記す。

\section{分化現象}

まさに分化現象こそが「多体系」を可能にしているのではないか。すなわち、たとえばおなじ種類の粒子が「惹かれ合う」状況を考えると、いくつかの粒子を混ぜ合わせた場を与えたとして、時間の果てには、これらは「分化」している。

局所的に粒子が「分化」してほしいとすれば、大域的にも「分化」した解が生まれ、また局所的に粒子が「混合」してほしいとすれば、大域的にも「混合」する。このような「微分方程式論」的な「もののことわり」によって、おのおのの粒子の物性による「大域的な解」が生まれる - このような方法によって「多体系」が可能となっているのではないか。さらに、「分化」している、すなわち「極限的な解である」がゆえに、「大域的性質」というものが可能となっているのではないか。

「極限的な状態」であるというのは、「保存量」が生じているということであるが、「ある多体系がなぜ保存量によって大域的に統制できるのか」という疑問について - まさしく「その解が極限的な状態であるから」という理解 (解釈) が可能であろう。

\begin{ex}[剛体力学]
剛体力学とは、すなわち対象とするオブジェが「剛体」であって、ある種に極限的な状態である。このような剛体の系について、これらの挙動はまさに「大域的な剛体の不変量」によって統制される。「非剛体」を剛体力学で理解することはむずかしいであろうことも想像されたい。
\end{ex}

すなわち、解が「あまりにも極限的である」ことを要求されるがゆえに、結局のところ「保存量」にデータを reduct することが「おおきな誤差なく可能となってしまう」ことに「熱力学の可能」の源泉があるのではないか - と憶測している。

\section{相転移}

しばしば「多体系」は相転移をおこなう。そもそも「大域的挙動」というのが「極限的な事項」であるがゆえに、その挙動は「分化」する。

\begin{ex}
現世物理でいえば、水などはしばしば相転移する。
\end{ex}

極限的な状態のあいだをうつりかわる - という「大域的な認知」のもとでは、局所的な物性がある閾値を超えたときに、系が「別物に化ける」ことがある。この「相転移」は、たとえば「認知機構」においても実際に起こっているのではないかと推察される。

\begin{ex}
実数 $r$ について、$r^t$ の $t \to \infty$ における挙動は、「分化」している。これもある種には相転移の例であるといえよう - ただしこの例はもっとも単純な場合である。またこの場合、$0$, $1$, $\infty$ は「結晶」のようにおもえる。
\end{ex}

「カオス」とは、この相転移現象が「起こりやすい」ことを指すのではないか ?

数の原理、マクロの原理によって、結晶状態のみが基本的に (認知のレベルにおいては) 議論の対象となり、それらのあいだの相関をみることが「認知」のレベルの現象理解には重要である - と考えられる。

\begin{note}
ここまでの理解を呪術に応用できないか - たとえば、「自然状態」においてはまさにさきほどのべたようなかたちで「世界場が計算を実行」し、「結晶状態」を実現している。しかしながら、この結晶状態を「操作によって乱す」ことは可能である。ドレッシングを「振る」ことによって、「水と油を混ぜる」ことが可能となる。あるいは、「触媒を用意する」ことによって、そこでは発生しなかった「化学反応」をひきおこすこともできる。こういったかたちによって「操作」をより「概念的に理解」することができるのではないか ?
\end{note}

\section{展望}

いくつかの今後考えたいトピックについて、以下に羅列する。

\begin{itemize}
\item 認知の「世界場へのチューニング機能」
\item ポリティクスの positivity
\item 情報理論
\item 内的瞑想
\item 「操作」とはなにか ?
\item 「数学的」文化人類学
\end{itemize}

\end{document}
















